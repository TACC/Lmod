\documentclass[12pt]{article}
%\usepackage{diss_inc}
%\usepackage{psfig}
\usepackage{underscore}
\setlength{\textwidth}{6.5in}
\setlength{\textheight}{9.0in}
\setlength{\topmargin}{0.0in}
\setlength{\headheight}{0.0in}
\setlength{\headsep}{0.0in}
\setlength{\oddsidemargin}{0in}
\setlength{\evensidemargin}{0in}
\begin{document}

Here are the major points for the SC 11 paper.

\begin{itemize}
  \item Software packages are being updated all the time.
    \begin{itemize}
      \item Users need access to new versions for features and bug
        fixes and older versions for stability.
      \item Let Users choose which version they want to use.
      \item Use /opt/apps/PACKAGE/VERSION style placement for optional
        software 
    \end{itemize}

  \item Use a Package manager for all optional software.
    \begin{itemize}
      \item Makes it easy to ``shoot'' nodes.
      \item Capture institutional knowledge in ``spec'' files.
    \end{itemize}

  \item Use Environment Modules to manage access to packages.
    \begin{itemize}
      \item Users want access to a resource or package and do not care
        where that package exists in the directory structure.
      \item Module makes it easy for users to choose which version of
        a package to use.
      \item Provide users with a default set of modules.
    \end{itemize}

  \item Hierarchy
    \begin{itemize}
      \item Application and pre-built libraries must be compiled and
        linked with same compiler.
      \item MPI based parallel libraries must be built and linked with
        the same MPI-compiler pairing.
      \item Organize Module files in a hierarchy rather than a flat system.
      \item Flat would have names like:
        \texttt{parallelLib\_1.2-mpiLib\_2.4-compiler\_12.2}
      \item Hierarchical layout means that users can only see the
        modules that are valid for the current MPI-compiler pairing.
      \item Flat system shows ALL modules that available with no guidiance.
    \end{itemize}


  \item Use a new implementation of Env. Modules $\rightarrow$ Lmod.

  \item Important Lmod features
    \begin{itemize}
      \item Users can not have two modules with the first name loaded
        at the same time.
        (i.e. users can not load both gcc/4.4.1 and gcc/4.5.2)
      \item When users swap compilers, Lmod automatically unload
        modules dependent on the old compiler and loads the same
        modules dependent on the new compiler.
      \item If any module is not available for the new compiler,
        it is marked inactive.  Everytime the MODULEPATH is changed
        the inactive modules are tested to see if they can be made active.
      \item Users are protected from loading two compilers or two MPI
        libraries at the same time by a new module directive.
      \item ``module spider''
    \end{itemize}


  \item Recording the module used by users.
    \begin{itemize}
      \item Find out which modules are not used or lightly used for
        safe removal.
      \item For lightly used modules, send e-mail to just those users.
      \item Tell users of one module that another module can perform
        better. 
    \end{itemize}

\end{itemize}



\end{document}
