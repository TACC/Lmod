\documentclass{beamer}

% You can also use a 16:9 aspect ratio:
%\documentclass[aspectratio=169]{beamer}
\usetheme{TACC16}

% It's possible to move the footer to the right:
%\usetheme[rightfooter]{TACC16}

%% page 
%\begin{frame}{}
%  \begin{itemize}
%    \item
%  \end{itemize}
%\end{frame}
%
%% page 
%\begin{frame}[fragile]
%    \frametitle{}
% {\tiny
%    \begin{semiverbatim}
%    \end{semiverbatim}
%}
%  \begin{itemize}
%    \item
%  \end{itemize}
%
%\end{frame}

\begin{document}
\title[Lmod]{Lmod tools to check module syntax}
\author{Robert McLay} 
\date{February 14, 2023}

% page 1
\frame{\titlepage} 


% page 2
\begin{frame}{Outline}
  \center{\includegraphics[width=.9\textwidth]{Lmod-4color@2x.png}}
  \begin{itemize}
    \item How to check module syntax when building an RPM or other
      software packages
    \item How to check your sites' entire module tree.
  \end{itemize}
\end{frame}

% page 3
\begin{frame}{Checking Modulefile Syntax}
  \begin{itemize}
    \item We want to check the syntax
    \item Not the action!
    \item We ignore the actions of loads(), depends\_on() ...
    \item How to evaluate all the syntax but not the action?
  \end{itemize}
\end{frame}

% page 4
\begin{frame}{Quick Review: How Lmod evaluates modulefile functions}
  \begin{itemize}
    \item Remember that commands like \texttt{setenv("ABC","3")} do
      different things
    \item On module load it sets \texttt{ABC} to \texttt{3}
    \item On module unload it unsets \texttt{ABC}
    \item On module show it prints \texttt{setenv("ABC","3")}
  \end{itemize}
\end{frame}

% page 5
\begin{frame}[fragile]
    \frametitle{src/modfuncs.lua: setenv()}
  \begin{itemize}
    \item Almost all modulefile function work like setenv()
    \item They check the syntax
    \item Then they call a member function in the MainControl Class
    \item The \texttt{mcp} variable is global MainControl object
    \item \texttt{mcp} controls what mode Lmod is in.
  \end{itemize}
 {\tiny
    \begin{semiverbatim}
function setenv(...)
   if (not l_validateArgsWithValue("setenv",...)) then 
     return 
   end
   mcp:setenv(...)
   return
end
    \end{semiverbatim}
}
\end{frame}

% page 6
\begin{frame}{Controlling Lmod Evaluation Mode}
  \begin{itemize}
    \item The \texttt{mcp} object acts as a big switchboard.
    \item When loading: \texttt{mcp} constructs derived class
      MC\_Load object.
    \item When unloading: \texttt{mcp} constructs derived class
      MC\_Unload object.
    \item When checking syntax:\texttt{mcp} constructs derived class
      MC\_CheckSyntax object.
  \end{itemize}
\end{frame}



% page 7
\begin{frame}[fragile]
    \frametitle{src/MC\_Load.lua}
 {\tiny
    \begin{semiverbatim}
M.load                 = MainControl.load
M.myModuleName         = MainControl.myModuleName
M.prepend\_path         = MainControl.prepend\_path
M.prereq               = MainControl.prereq
M.setenv               = MainControl.setenv
M.set\_alias            = MainControl.set\_alias
M.unload               = MainControl.unload
M.unsetenv             = MainControl.unsetenv
    \end{semiverbatim}
}
  \begin{itemize}
    \item These are the normal mapping when loading
  \end{itemize}

\end{frame}

% page 8
\begin{frame}[fragile]
    \frametitle{src/MC\_Unload.lua}
 {\tiny
    \begin{semiverbatim}
M.load                 = MainControl.unload
M.myModuleName         = MainControl.myModuleName
M.prepend\_path         = MainControl.remove\_path\_first
M.prereq               = MainControl.quiet
M.setenv               = MainControl.unsetenv
M.set\_alias            = MainControl.unset\_alias
M.unload               = MainControl.unload
M.unsetenv             = MainControl.quiet
    \end{semiverbatim}
}
  \begin{itemize}
    \item These are the normal mapping when unloading
  \end{itemize}
\end{frame}

% page 9
\begin{frame}[fragile]
    \frametitle{src/MC\_CheckSyntax.Lua}
 {\tiny
    \begin{semiverbatim}
M.load                 = MainControl.load
M.myModuleName         = MainControl.myModuleName
M.prepend\_path         = MainControl.prepend\_path
M.prereq               = MainControl.quiet
M.setenv               = MainControl.setenv
M.set\_alias            = MainControl.set\_alias
M.unload               = MainControl.quiet
M.unsetenv             = MainControl.unsetenv
    \end{semiverbatim}
}
  \begin{itemize}
    \item See that the actions of prereq() are ignored
    \item This way actions outside a module are ignored
    \item The syntax of all commands are checked.
    \item \texttt{MainControl.load} is a special case
  \end{itemize}

\end{frame}

% page 10
\begin{frame}[fragile]
    \frametitle{\texttt{MainControl.load} is a special case}
  \begin{itemize}
    \item Loading the module under test must use it
    \item So load\_usr()  function has this test
    \item This prevent module loading other modules
    \item Because the frameStk count will be $>$ 1.
  \end{itemize}
 {\tiny
    \begin{semiverbatim}
function M.load_usr(self, mA)
   local frameStk = FrameStk:singleton()
   if (checkSyntaxMode() and frameStk:count() > 1) then
      return {}
   end

   l_registerUserLoads(mA)
   local a = self:load(mA)
   return a
end
    \end{semiverbatim}
}

\end{frame}

% page 11
\begin{frame}[fragile]
    \frametitle{\texttt{module --checkSyntax load bad/1.0}}
 {\tiny
    \begin{semiverbatim}
\$ cat bad/1.0.lua
setenv("ONE")

\$ module --checkSyntax load bad   
Lmod has detected the following error:  setenv("ONE") is not valid;
a value is required. 
While processing the following module(s):
    Module fullname  Module Filename
    ---------------  ---------------
    bad/1.0          /home/user/myModules/bad/1.0.lua

    \end{semiverbatim}
}
  \begin{itemize}
    \item We here at TACC use this when building module files in an RPM
      *.spec file
    \item Other build tools might find this useful
  \end{itemize}
\end{frame}


% page 12
\begin{frame}{New command check\_module\_tree\_syntax}
  \begin{itemize}
    \item I wanted a way to check every modulefile in a site's tree.
    \item The command \texttt{module spider} used to do that. 
    \item But sending site error to user is not a good idea.
  \end{itemize}
\end{frame}

% page 13
\begin{frame}{New command check\_module\_tree\_syntax (II)}
  \begin{itemize}
    \item Site wanted to know when module directory had multiple files
      marking a default
    \item There are upto 4 ways to do this in a modulefile directory
      \begin{enumerate}
        \item a default symlink
        \item a .modulerc.lua file
        \item a .modulerc file
        \item a .version file
      \end{enumerate}
    \item The priority is in this order.
  \end{itemize}
\end{frame}

% page 14
\begin{frame}{History}
  \begin{itemize}
    \item The site wanted to report to the user when this happened.
    \item I try hard not to report site errors to users.
    \item After all what are they suppose to do with this info?
    \item So I modified \texttt{\$LMOD\_DIR/spider} to walk the module
      tree.
    \item To make \texttt{\$LMOD\_DIR/check\_module\_tree\_syntax}
    \item This reports both syntax error/warning and duplicate marked
      default files in a modulefile directory.
    \item This command uses the MC\_CheckSyntax mode to eval each modulefiles.
  \end{itemize}
\end{frame}

% page 15
\begin{frame}{Changes to spider cache format}
  \begin{itemize}
    \item Added defaultA to remember all marked defaults in directory
    \item This array is process to use the highest priority marked
      default file.
    \item The old format just kept the highest priority marked default
      file in defaultT.
  \end{itemize}
\end{frame}

% page 16
\begin{frame}[fragile]
    \frametitle{Example results}
 {\tiny
    \begin{semiverbatim}
The following directories have more than one marked default file:
-----------------------------------------------------------------
  /home/user/w/lmod/rt/ck_mtree_syntax/mf/A


The following modulefile(s) have syntax errors:
-----------------------------------------------
  ModuleName: A/1.0, Fn: /mf/A/1.0.lua 
     Error: [string "setenv("MY_VERSION", myModuleVersion())..."]:2: unexpected symbol near '#' 
  ModuleName: A/2.0, Fn: /mf/A/2.0 
     Error: /mf/A/2.0: (A/2.0): invalid command name "nonExistantCmd"
  ModuleName: hashrf/6.0.1, Fn: /mf/hashrf/6.0.1.lua 
     Error: [string "local name = "hashrf"..."]:16: attempt to call a nil value (global 'WTF') 
  ModuleName: papi/4.4.0, Fn: /mf/papi/4.4.0.lua 
     Error: command: help, one or more arguments are not strings. 
    \end{semiverbatim}
}
  \begin{itemize}
    \item
  \end{itemize}

\end{frame}

% page 17
\begin{frame}{Conclusions}
  \begin{itemize}
    \item Two ways to check for modulefile syntax errors.
    \item One for building modulefiles
    \item Another for checking site module tree.
  \end{itemize}
\end{frame}

% page 18
\begin{frame}{Future Topics}
  \begin{itemize}
    \item Unknown at the moment.
    \item Next Meeting will be March 7th at 9:30 Central (15:30 UTC)
  \end{itemize}
\end{frame}

\end{document}
