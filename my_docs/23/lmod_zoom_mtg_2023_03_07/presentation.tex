\documentclass{beamer}

% You can also use a 16:9 aspect ratio:
%\documentclass[aspectratio=169]{beamer}
\usetheme{TACC16}

% It's possible to move the footer to the right:
%\usetheme[rightfooter]{TACC16}

%% page 
%\begin{frame}{}
%  \begin{itemize}
%    \item
%  \end{itemize}
%\end{frame}
%
%% page 
%\begin{frame}[fragile]
%    \frametitle{}
% {\tiny
%    \begin{semiverbatim}
%    \end{semiverbatim}
%}
%  \begin{itemize}
%    \item
%  \end{itemize}
%
%\end{frame}

\begin{document}
\title[Lmod]{Using Personal Modules and Inherit() w/ the Software Hierarchy}
\author{Robert McLay} 
\date{March 7, 2023}

% page 1
\frame{\titlepage} 


% page 2
\begin{frame}{Outline}
  \center{\includegraphics[width=.9\textwidth]{Lmod-4color@2x.png}}
  \begin{itemize}
    \item How to correctly use personal modules so that Lmod will find
      and use them.
    \item How to setup a personal library/application in the software
      hierarchy
    \item Why this is a PITA!
  \end{itemize}
\end{frame}

% page 3
\begin{frame}{Creating Personal modules}
  \begin{itemize}
    \item What is the big deal?
    \item Easy: Create a directory (say \$HOME/my\_modules)
    \item Create a directory (\$HOME/my\_modules/acme)
    \item Create a create modulefile: \$HOME/my\_modules/acme/3.2.lua)
    \item \$ \texttt{module use \$HOME/my\_modules}
    \item \$ \texttt{module load acme/3.2}
    \item Easy right!?
  \end{itemize}
\end{frame}

% page 4
\begin{frame}{Testing a personal copy of a system module}
  \begin{itemize}
    \item Suppose that acme/3.2 is already on your system
    \item And acme/3.2 is a marked default
    \item The command \texttt{module load acme/3.2}
    \item Will load the system one and not yours
    \item Even though \$HOME/my\_modules is listed first in \$MODULEPATH
    \item Why?
  \end{itemize}
\end{frame}

% page 5
\begin{frame}[fragile]
    \frametitle{Why?}
  \begin{itemize}
    \item While Lmod does look in \$MODULEPATH order
    \item So the first module found is usually picked.
    \item {\color{red}\emph{However}}, Marked defaults ALWAYS win in
      N/V module layouts (Best found)
    \item Note not in N/V/V layouts. (First found)
    \item A marked default is where there is either:
      \begin{enumerate}
        \item A default symlink
        \item .modulerc.lua
        \item .modulerc
        \item .version
      \end{enumerate}
    \item I recently updated the documentation
      https://lmod.readthedocs.io/en/latest/060\_locating.html
      to explain this
  \end{itemize}
\end{frame}

% page 6
\begin{frame}[fragile]
    \frametitle{Getting Around a System Marked Defaults}
  \begin{itemize}
    \item Make your own marked default.
    \item Easiest way is to make a default symlink
  \end{itemize}
 {\tiny
    \begin{semiverbatim}
        \$ cd \$HOME/my\_modules/acme
        \$ ln -s 3.2.lua default
    \end{semiverbatim}
}
\end{frame}



% page 7
\begin{frame}[fragile]
    \frametitle{Checking with module avail}
 {\tiny
    \begin{semiverbatim}
   ---------- /home/user/my\_modules -----------
   acme/3.2 (D) 
 
   ---------- /opt/apps/modulefiles ------------
   StdEnv    acme/3.2

    \end{semiverbatim}
}
  \begin{itemize}
    \item Make sure that the (D) is next to your acme module 
    \item And not the system one.
  \end{itemize}
\end{frame}


% page 8
\begin{frame}{Bigger issue: Testing a compiler dependent \texttt{boost/1.85.0}}
  \begin{itemize}
    \item And you want it part of the software hierarchy
    \item How can you do this \texttt{without} modifying the system
      modulefiles?
    \item In particular you only want the correct version of boost
      available when you load the correct compiler.
  \end{itemize}
\end{frame}

% page 9
\begin{frame}{The short answer: inherit()}
  \begin{itemize}
    \item You can use the inherit() function to simplify this a little
    \item This is discussed in detail in https://lmod.readthedocs.io/en/latest/340\_inherit.html
  \end{itemize}
\end{frame}

% page 10
\begin{frame}{Overview}
  \begin{itemize}
    \item We want to test/use boost install from our own account.
    \item And have it load when the ``right'' compiler is loaded
    \item This assumes that your site is using the software hierarchy
    \item How can we get the system compiler to load our directory
      into \$MODULEPATH?
    \item Suppose we want to test a boost version with the intel 19.1
      and gcc 12.2 compilers
  \end{itemize}
\end{frame}

% page 11
\begin{frame}{Building and Installing boost in your account}
  \begin{itemize}
    \item Build boost 1.85.0 with gcc 12.2 $\Rightarrow$ \textasciitilde/pkg/gcc-12/boost/1.85.0
    \item Build boost 1.85.0 with intel 19.1 $\Rightarrow$ \textasciitilde/pkg/intel-19/boost/1.85.0
  \end{itemize}
\end{frame}

% page 12
\begin{frame}{Choice 1: Copy compiler modules into your account}
  \begin{itemize}
    \item Easy to do
    \item Add your directory into \$MODULEPATH
    \item Problem: you are now responsible to keep your copy up-to-date
  \end{itemize}
\end{frame}

% page 13
\begin{frame}{Choice 2: Use inherit()}
  \begin{itemize}
    \item Create your own compiler module and inherit from the system
      one.
    \item The inherit() function take NO arguments
    \item Lmod looks for the exact same name in \$MODULEPATH
    \item This way it includes the system one with your changes.
  \end{itemize}
\end{frame}





% page 18
\begin{frame}{Conclusions}
  \begin{itemize}
    \item Two ways to check for modulefile syntax errors.
    \item One for building modulefiles
    \item Another for checking site module tree.
  \end{itemize}
\end{frame}

% page 19
\begin{frame}{Future Topics}
  \begin{itemize}
    \item Unknown at the moment.
    \item Next Meeting will be March 7th at 9:30 Central (15:30 UTC)
  \end{itemize}
\end{frame}

\end{document}
