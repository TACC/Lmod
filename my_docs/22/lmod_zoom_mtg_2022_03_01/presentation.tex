\documentclass{beamer}

% You can also use a 16:9 aspect ratio:
%\documentclass[aspectratio=169]{beamer}
\usetheme{TACC16}

% It's possible to move the footer to the right:
%\usetheme[rightfooter]{TACC16}

\begin{document}
\title[Lmod]{Lmod Testing System}
\author{Robert McLay} 
\date{March. 1, 2022}

% page 1
\frame{\titlepage} 


% page 2
\begin{frame}{Lmod Testing System}
  \center{\includegraphics[width=.9\textwidth]{Lmod-4color@2x.png}}
  \begin{itemize}
    \item Testing philosophy in Lmod
    \item Goals of testing Lmod
    \item Hermes/tm basic operations
    \item Details of how an Lmod test works
    \item Future Topics
  \end{itemize}
\end{frame}

% page 3
\begin{frame}{Testing philosophy in Lmod}
  \begin{itemize}
    \item Lmod's success relies heavily on the testing system.
    \item Passing all the tests usually means a new version can be released.
    \item I don't think that anyone is using it beside Lmod (But it is
      very useful)
    \item My philosophy is to test features in general
    \item Not to setup a torture test
    \item No way I can test every possible scenario.
    \item My imagination is not that good.
  \end{itemize}
\end{frame}


% page 4
\begin{frame}{Goals of testing Lmod}
  \begin{itemize}
    \item Test various features of Lmod.
    \item New feature won't break old features.
    \item Test Lmod on Linux/MacOS, Lua 5.1 to 5.4
    \item Make development of Lmod easier.
    \item Add tests of new bugs $\Rightarrow$ Don't repeat them!
  \end{itemize}
\end{frame}

% page 5
\begin{frame}{It is hard to test everything}
  \begin{itemize}
    \item Testing Old data with new versions(Collections, spiderT.lua)
    \item One test (end2end) builds Lmod and tests the built version
    \item All other tests use the source code directly
    \item Special hacks to use configuration options. 
    \item Environment variable are checked NOT configuration options
  \end{itemize}
\end{frame}

% page 6
\begin{frame}{Hermes/tm Testing system}
  \begin{itemize}
    \item Hermes is a group of tools to help with testing
    \item tm is the testing manager.
    \item The main function of tm is to select tests and run them.
    \item Each test is independent!
    \item tm knows \emph{nothhing} about what is being tested.
    \item Must tell if test passed via special file (Lua file named
      t1.results)
    \item Three kinds of results
      \begin{enumerate}
        \item Passed: All steps passed
        \item Failed: Did not produce a t1.results file
        \item Diffed: Produced diffs between gold files and test
          result files.
      \end{enumerate}
  \end{itemize}
\end{frame}

% page 7
\begin{frame}{\texttt{tm} flow}
  \begin{itemize}
    \item \texttt{tm} searches for tests from the current directory
      down
    \item It is looking for files with the *.tdesc extension (testDir)
    \item Once all tests have been selected, it runs them all
    \item For each test directory a sub-dir tree is created.
    \item Typically: t1/$<$\$TARG$>$-$<$date\_time$>$-$<$uname
      -s$>$-$<$arch$>$-$<$test\_name$>$
    \item The above dir is the outputDir
    \item The test is run in \$outputDir
  \end{itemize}
\end{frame}

% page 8 
\begin{frame}{Every project using tm must have a \emph{diff} tool}
  \begin{itemize}
    \item There must be an automatic way to decide a test passed.
    \item A numerical code can use an $L^2$ norm. 
    \item The new answer can be different but close w/ numerical codes.
    \item Lmod use diff on stdout and stderr between gold and test
      results
    \item Filtering is required to deal with OS and file location
      differences
    \item To pass the filtered result {\color{blue} \emph{must}} be
      the same.
  \end{itemize}
\end{frame}

% page 9
\begin{frame}{*.tdesc file key-value pairs}
  \begin{itemize}
    \item The testDescript is a parameterized bash script
    \item Some special parameters are:
      \begin{enumerate}
        \item \$(testDir): where the *.tdesc is located
        \item \$(projectDir): where Hermes.db is located (top of the
          project)
        \item \$(outputDir): where the test is run
        \item \$(resultFn): The name of the results lua file.
      \end{enumerate}
  \end{itemize}
\end{frame}



% page 9
\begin{frame}{Lmod tests}
  \begin{itemize}
    \item 
    \item 
  \end{itemize}
\end{frame}




% page 19
\begin{frame}{Future Topics}
  \begin{itemize}
    \item Write one new test.
    \item Explain how Mname object converts names into a filename.
    \item More internals of Lmod?
  \end{itemize}
\end{frame}

\end{document}
