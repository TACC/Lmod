\documentclass{beamer}

% You can also use a 16:9 aspect ratio:
%\documentclass[aspectratio=169]{beamer}
\usetheme{TACC16}

% It's possible to move the footer to the right:
%\usetheme[rightfooter]{TACC16}

\begin{document}
\title[Lmod]{Lmod: A Modern Environment Module System}

% page 1
\frame{\titlepage} 

% page 2
\begin{frame}{Outline}
  \begin{itemize}
    \item What are Environment Modules?
    \item What is Lmod?
    \item Can Lmod help manage your site?
    \item Advanced Topics
    \item Where to go for help.
  \end{itemize}
\end{frame}

% page 3
\begin{frame}{What are Modules?}
  \begin{itemize}
    \item Modules are the way sites provide optional software: MPI,
      Boost, ...
    \item Modules add to PATH and set other env. vars for each package
    \item Modules can be unloaded.
    \item Modulefiles are in one file not one for each
      shell. (e.g. Compiler init scripts)
    \item It is one of the main way we communicate with our users.
    \item Main Function: Change User Env, Modify Path-like vars, Add/Remove
      aliases, ...
  \end{itemize}
\end{frame}

% page 4
\begin{frame}{What is Lmod?}
  \begin{itemize}
    \item A modern replacement for a tried and true concept.
    \item The guiding principal: ``Make life easier without getting in
      the way.''
    \item Reads both TCL and Lua modulefiles
  \end{itemize}
\end{frame}

% page 5
\begin{frame}{Fundamental Issues}
  \begin{itemize}
    \item Software Packages are created and updated all the time.
    \item Some Users need new versions for new features and bug fixes.
    \item Other Users need older versions for stability and continuity.
    \item No system can support all versions of all packages.
    \item User programs using pre-built C++ \& Fortran libraries must link with the same compiler.
    \item Similarly, MPI Applications must build and link with same
      MPI/Compiler pairing when using pre-built MPI libraries.
  \end{itemize}
\end{frame}

% page 6
\begin{frame}[fragile]
    \frametitle{Example of Lmod: Environment Modules (I)}
    {\tiny
\begin{semiverbatim}
\$ {\color{blue} module avail}
------------------ /opt/apps/modulefiles/MPI/intel/12.0/mpich2/1.4 ------------------
  petsc/3.1 (D)    petsc/3.1-debug    pmetis/4.0    tau/2.20.3

------------------- /opt/apps/modulefiles/Compiler/intel/12.0 -----------------------
  boost/1.45.0        gotoblas2/1.13      openmpi/1.4.3
  boost/1.46.0        mpich2/1.3.2        openmpi/1.5.1
  boost/1.46.1 (D)    mpich2/1.4    (D)   openmpi/1.5.3   (D)

-------------------------- /opt/apps/modulefiles/Core -------------------------------
  StdEnv               intel/11.1         papi/4.1.4
  admin/admin-1.0      intel/12.0  (D)    scite/2.28
  ddt/ddt              lmod/lmod          tex/2010
  dmalloc/dmalloc      local/local (D)    unix/unix    (D)
  fdepend/1.2          mkl/mkl            visit/visit
  gcc/4.4              noweb/2.11b
  gcc/4.5        (D)
\end{semiverbatim}
    }
\end{frame}

% page 7
\begin{frame}[fragile]
    \frametitle{Example of Lmod: Environment Modules (II)}
    {\tiny
\begin{semiverbatim}
{\color{blue}\$ module list}
Currently Loaded Modules:
  1) StdEnv  2) gcc/4.5  3) mpich2/1.4  4) petsc/3.1
{\color{blue}\$ module unload gcc}
Inactive Modules:
  1) mpich2  2) petsc
{\color{blue}\$ module list}
Currently Loaded Modules:
  1) StdEnv
Inactive Modules:
  1) mpich2  2) petsc
{\color{blue}\$ module load intel}
Activating Modules:
  1) mpich2  2) petsc
{\color{blue}\$ module swap intel gcc}
Due to MODULEPATH changes the follow modules have been reloaded:
  1) mpich2  2) petsc
\end{semiverbatim}
    }
\end{frame}

% page 8
\begin{frame}{Can tools like Lmod improve the user experience?}
  \begin{itemize}
    \item Sites provide packages: applications and libraries
    \item Users can pick which packages and version to suit their needs
    \item But what we are really after is to cut down on tickets!
    \item We also want to make it difficult for users to load
      incompatible modules
  \end{itemize}
\end{frame}




\end{document}
