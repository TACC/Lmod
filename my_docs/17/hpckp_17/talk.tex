\documentclass{beamer}

% You can also use a 16:9 aspect ratio:
%\documentclass[aspectratio=169]{beamer}
\usetheme{TACC16}

% It's possible to move the footer to the right:
%\usetheme[rightfooter]{TACC16}

\begin{document}
\title[Lmod]{Lmod: A Modern Environment Module System}

% page 1
\frame{\titlepage} 

% page 2
\begin{frame}{Outline}
  \begin{itemize}
    \item What are Environment Modules?
    \item What is Lmod?
    \item Can Lmod help manage your site?
    \item Advanced Topics
    \item Where to go for help.
  \end{itemize}
\end{frame}

% page 3
\begin{frame}{What are Modules?}
  \begin{itemize}
    \item Modules help user load the SW packages they need: MPI, Boost, ...
    \item Modules can be unloaded.
    \item Modulefiles are in one file not one for each
      shell. (e.g. compiler init scripts)
    \item It is one of the main ways we communicate with our users.
    \item The main function of Lmod is to change the user environment
    \item It modifies PATH, LD_LIBRARY_PATH, ...
    \item It sets other environment variables and defines aliases
  \end{itemize}
\end{frame}

% page 4
\begin{frame}{What is Lmod?}
  \begin{itemize}
    \item A modern replacement for a tried and true concept.
    \item The guiding principal: ``Make life easier without getting in
      the way.''
    \item Reads both TCL and Lua modulefiles
  \end{itemize}
\end{frame}

% page 5
\begin{frame}{Fundamental Issues}
  \begin{itemize}
    \item Software Packages are created and updated all the time.
    \item Some Users need new versions for new features and bug fixes.
    \item Other Users need older versions for stability and continuity.
    \item No system can support all versions of all packages.
    \item User programs using pre-built C++ \& Fortran libraries must link with the same compiler.
    \item Similarly, MPI Applications must build and link with same
      MPI/Compiler pairing when using pre-built MPI libraries.
  \end{itemize}
\end{frame}


% page 6
\begin{frame}[fragile]
    \frametitle{Example of Lmod: Environment Modules (I)}
    {\tiny
\begin{semiverbatim}
{\color{blue}\$ module list}
Currently Loaded Modules:
  1) StdEnv  2) gcc/4.5  3) mpich2/1.4  4) petsc/3.1
{\color{blue}\$ module unload gcc}
Inactive Modules:
  1) mpich2  2) petsc
{\color{blue}\$ module list}
Currently Loaded Modules:
  1) StdEnv
Inactive Modules:
  1) mpich2  2) petsc
{\color{blue}\$ module load intel}
Activating Modules:
  1) mpich2  2) petsc
{\color{blue}\$ module swap intel gcc}
Due to MODULEPATH changes the follow modules have been reloaded:
  1) mpich2  2) petsc
\end{semiverbatim}
    }
\end{frame}

% page 7
\begin{frame}[fragile]
    \frametitle{Example of Lmod: Environment Modules (II)}
    {\tiny
\begin{semiverbatim}
\$ {\color{blue} module avail}
------------------ /opt/apps/modulefiles/MPI/intel/12.0/mpich2/1.4 ------------------
  petsc/3.1 (D)    petsc/3.1-debug    pmetis/4.0    tau/2.20.3

------------------- /opt/apps/modulefiles/Compiler/intel/12.0 -----------------------
  boost/1.45.0        gotoblas2/1.13      openmpi/1.4.3
  boost/1.46.0        mpich2/1.3.2        openmpi/1.5.1
  boost/1.46.1 (D)    mpich2/1.4    (D)   openmpi/1.5.3   (D)

-------------------------- /opt/apps/modulefiles/Core -------------------------------
  StdEnv               intel/11.1         papi/4.1.4
  admin/admin-1.0      intel/12.0  (D)    scite/2.28
  ddt/ddt              lmod/lmod          tex/2010
  dmalloc/dmalloc      local/local (D)    unix/unix    (D)
  fdepend/1.2          mkl/mkl            visit/visit
  gcc/4.4              noweb/2.11b
  gcc/4.5        (D)
\end{semiverbatim}
    }
\end{frame}

% page 8
\begin{frame}{Can tools like Lmod improve the user experience?}
  \begin{itemize}
    \item Sites provide packages: applications and libraries
    \item Users can pick which packages and version to suit their needs
    \item But what we are really after is to cut down on tickets!
    \item Or simply make your resources easier for users to use.
  \end{itemize}
\end{frame}

% page 9
\begin{frame}{Lmod examples}
  \begin{itemize}
    \item Lmod was first released in 2009
    \item It is the only module system used at TACC since 2010
    \item The following are some examples of how Lmod can help
  \end{itemize}
\end{frame}

% page 10
\begin{frame}{Can Lmod help with the /usr/local/bin problem?}
  \begin{itemize}
    \item Suppose your startup files put /usr/local/bin in PATH
    \item And suppose module BAR also adds /usr/local/bin to PATH
    \item Currently Loading then unloading BAR will remove
      /usr/local/bin from PATH. 
    \item Site can configure Lmod to support duplicate paths
    \item Or coming soon Lmod will support reference counting!
  \end{itemize}
\end{frame}

% page 11
\begin{frame}{Can Lmod prevent users from mixing modules they shouldn't?}
  \begin{itemize}
      \item Same Name modules:
      \begin{itemize}
        \item Things can get confusing when users load two gcc modules
        \item Normally, Lmod will unload old gcc, then load new gcc
        \item Optionally, sites can auto-conflict with themselves.
      \end{itemize}
    \item Loading two compilers or MPI Stack:
      \begin{itemize}
        \item It is a rare user who needs to load two MPI stacks or
          two different compilers
        \item GCC and Intel are a special case.
        \item Sites can add family("compiler") to compiler modules
        \item This will autoswap one compiler for another!
        \item Similarly for MPI modules.
      \end{itemize}
  \end{itemize}
\end{frame}

% page 12
\begin{frame}{How to manage software: New or Old}
  \begin{itemize}
    \item How can you test new/experimental software?
    \item Suppose your site keeps SW for the life of machine?
    \item How do you encourage usage of newer SW w/o breaking old job
      scripts?
    \item Lmod now supports hiding regular modules from avail and
      spider.
    \item Hidden modules can still be loaded.
    \item Modules can be explicitly marked as hidden
    \item Or you can use the isVisible hook
    \item Both sites and users can hide modules
  \end{itemize}
\end{frame}

% page 13
\begin{frame}{Can Lmod help with deprecating packages?}
  \begin{itemize}
    \item Suppose your site keeps a limited number of versions (say 3
      or less)
    \item How to you decide which package to keep or remove?
    \item Lmod support optional tracking of what packages are loaded
      by whom.
    \item You can send targeted email to those users about
      deprecation based on tracking.
    \item Independent of tracking: nag messaging
    \item Do not need to change modulefile!
    \item Users get a message when they load a deprecated module. 
  \end{itemize}
\end{frame}

% page 14
\begin{frame}{A site that does not want module defaults}
  \begin{itemize}
    \item Suppose your site produces Weather Forecasts or Processes
      Satellite Images.
    \item No one set of compilers etc will satify your needs.
    \item Site can set LMOD\_EXACT\_MATCH$=$yes $\Rightarrow$ There are no defaults
    \item Users \emph{MUST} specify name and version!
  \end{itemize}
\end{frame}

% page 15
\begin{frame}{Can users have their own default list of modules?}
  \begin{itemize}
    \item It is common to provide a default list of modules
    \item 
  \end{itemize}
\end{frame}





% page 15
\begin{frame}{Can Lmod deal with shared home file system?}
  \begin{itemize}
    \item Suppose 
  \end{itemize}
\end{frame}

% page 15
\begin{frame}{Can users easily grep the output from Lmod?}
  \begin{itemize}
    \item 
  \end{itemize}
\end{frame}

% page 16
\begin{frame}{Can Lmod work with non-english speakers?}
  \begin{itemize}
    \item 
  \end{itemize}
\end{frame}

% page 17
\begin{frame}{Can Lmod help with compiler and/or MPI/compiler
      depenedent modules?}
  \begin{itemize}
    \item 
  \end{itemize}
\end{frame}

% page 18
\begin{frame}{Can a Site control the output of module avail?}
  \begin{itemize}
    \item 
  \end{itemize}
\end{frame}

% page 19
\begin{frame}{Can Lmod help with software web pages?}
  \begin{itemize}
    \item 
  \end{itemize}
\end{frame}






%\item We also want to make it difficult for users to load
%incompatible modules.   



\end{document}
