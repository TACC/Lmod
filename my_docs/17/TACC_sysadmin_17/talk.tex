\documentclass[dvipsnames,aspectratio=169]{beamer}

\usepackage{etex}
\usepackage{tikz}
\usetikzlibrary{shapes.geometric, arrows}
\usetikzlibrary{decorations.pathreplacing}
\usetikzlibrary{calc,arrows.meta,positioning}
\usepackage{pstricks}
\usepackage{colortbl}
\usepackage{textcomp}
\usepackage{picture}
\usepackage{outlines}
\usepackage{listings}

\usepackage{./themes/beamerthemeTACC16-169}


\begin{document}
\title[Lmod]{Lmod: A Modern Environment System}
\author{Robert McLay} 
\date{April 20, 2017} 


% page 0
\frame{\titlepage} 

% page 1
\begin{frame}{Outline}
  \begin{itemize}
    \item What are Environment Modules?
    \item What is Lmod?
    \item Modulefiles Architecture
    \item Where to go for help.
  \end{itemize}
\end{frame}

% page 2
\begin{frame}{Conclusions: Lmod}
  \begin{itemize}
    \item Latest version: https://github.com:TACC/Lmod.git
    \item Stable version: http://lmod.sf.net
    \item Documentation:  http://lmod.readthedocs.org
    \item Mailing List:   lmod-users@lists.sourceforge.net.
    \item Join here: https://lists.sourceforge.net/lists/listinfo/lmod-users
  \end{itemize}
\end{frame}

% page 3
\begin{frame}{What are Modules?}
  \begin{itemize}
    \item Modules are the way sites provide optional software: MPI,
      Boost, ...
    \item Modules add to PATH and set other env. vars for each package
    \item Modules can be unloaded.
    \item Modulefiles are in one file not one for each
      shell. (e.g. Compiler init scripts)
    \item Demo
  \end{itemize}
\end{frame}

% page 4
\begin{frame}{What is Lmod?}
  \begin{itemize}
    \item A modern replacement for a tried and true concept.
    \item The guiding principal: ``Make life easier w/o getting in
      the way.''
    \item Reads both TCL and Lua modulefiles
  \end{itemize}
\end{frame}

% page 5
\begin{frame}{Why You Might Want To Use Lmod?}
  \begin{itemize}
    \item Same \texttt{module} command as in Tmod
    \item Active Development;  Frequent Releases; Bug fixes.
    \item Vibrant Community
    \item It is used from Norway to Isreal to New Zealand from Stanford to MIT to NASA
    \item Enjoy many capabilities w/o changing a single module file
    \item Debian, Fedora and Brew packages available
    \item Many more advantages when you're ready
    \item It is what we and many sites around the world use every day!
  \end{itemize}
\end{frame}

% page 6
\begin{frame}{Features}
  \begin{itemize}
    \item Reads for TCL and Lua modulefiles
    \item One name rule.
    \item Support a Software Hierarchy
    \item Fast \texttt{module avail} via optional spider cache 
    \item Properties (gpu, mic)
    \item Semantic Versioning:  5.6 is older than 5.10
    \item family(``compiler'') family(``mpi'') support
    \item Optional Tracking: What modules are used?
    \item Many other features: ml, collections, hooks, nag, ...
  \end{itemize}
\end{frame}

% page 7
\begin{frame}{Tmod vs. Lmod}
  \begin{itemize}
    \item Tmod is in maintenance mode, Lmod active
    \item Lmod has many more features
    \item Tmod: \texttt{module load gcc/5.3 gcc/6.0} works
    \item Lmod has the ``One Name Rule''
    \item Lmod close to Tmod, but not the same.
  \end{itemize}
\end{frame}

%page 8
\begin{frame}{Module Architecture Design}
  \begin{itemize}
    \item Optional software: shared or local?
    \item Flat or Hierarchical Module Layout?
    \item Naming conventions? (NV, CNV, NVV?)
    \item Keep software for life of cluster?
    \item Type of User: Expert, Novice, M\&S, Other?
    \item Conventions for contents of Modules?
  \end{itemize}
\end{frame}

%page 9
\begin{frame}{Shared Disk versus Local Install}
  \begin{itemize}
    \item Local install: Fast, small 
    \item Shared Disk: Big, slow
    \item TACC: local install (mostly)
    \item Limited to 3 version max!
    \item Shared Disk: Keep Software for life of system.
  \end{itemize}
\end{frame}

\begin{frame}{Modulefile Choices}
  \begin{itemize}
    \item Flat Naming Scheme
    \item Hierarchical Naming Scheme
  \end{itemize}
\end{frame}

\begin{frame}{Flat Naming Scheme: PETSc}
  PETSc is a parallel iterative solver package:
  \begin{itemize}
    \item Compilers: GCC 6.3, Intel 17.0
    \item MPI Implementations: MVAPICH2 2.1, IMPI 17.0
    \item MPI Solver package: PETSc 4.1
    \item 4 versions of PETSc: 2 Compilers $\times$ 2 MPI
  \end{itemize}
\end{frame}

\begin{frame}{Flat: PETSc }
  \begin{enumerate}
  \item \texttt{PETSc/4.1-mvapich2-2.1-gcc-6.3}
  \item \texttt{PETSc/4.1-mvapich2-2.1-intel-17.0}
  \item \texttt{PETSc/4.1-openmpi-1.8-gcc-6.3}
  \item \texttt{PETSc/4.1-openmpi-1.8-intel-17.0}
  \end{enumerate}
\end{frame}

\begin{frame}{Problems w/ Flat naming scheme}
  \begin{itemize}
    \item Users have to load modules:
      \begin{itemize}
        \item ``intel/17.0''
        \item ``mvapich2/2.1-intel-17.0''
        \item ``PETSc/4.1-mvapich2-2.1-intel-17.0''
        \item Changing compilers means unloading all three modules
        \item Reloading new compiler, MPI, PETSc modules.
        \item Not loading correct modules $\Rightarrow$ Mysterious Failures!
        \item Onus of package compatibility on users!
        \item Or extremely complicated modulefiles!
      \end{itemize}
  \end{itemize}
\end{frame}

\begin{frame}{Hierarchical Naming Schemes}
  \begin{itemize}
    \item Store modules under one tree: \texttt{/opt/apps/modulefiles}
    \item One strategy is to use sub-directories:
      \begin{itemize}
        \item Core: Regular packages: apps, compilers, git
        \item Compiler: Packages that depend on compiler: boost, MPI
        \item MPI: Packages that depend on MPI/Compiler: PETSc, TAU
      \end{itemize}

  \end{itemize}
\end{frame}

\begin{frame}{\texttt{MODULEPATH}}
  \begin{itemize}
    \item \texttt{MODULEPATH} is a colon separated list of directories
      containing directories and module files.
    \item No modulefiles loaded $\Rightarrow$ users can only load core modules.
    \item Loading a compiler module adds to the \texttt{MODULEPATH}:
      \begin{itemize}
        \item Users can load compiler dependent modules.
        \item This includes MPI implementations modules.
      \end{itemize}
    \item Loading an MPI module adds to the \texttt{MODULEPATH}:
      \begin{itemize}
        \item Users can load MPI libraries that match the MPI/compiler pairing.
      \end{itemize}
  \end{itemize}
\end{frame}


\begin{frame}{Hierarchical Examples: Core}
  \begin{itemize}
    \item Generic:
      \begin{itemize}
        \item Package: \texttt{/opt/apps/}\emph{package/version}
        \item M: {\color{blue}/opt/apps/modulefiles}
        \item Modulefile: \texttt{{\color{blue}\$M}/Core/}\emph{package/version}
      \end{itemize}
    \item Git 1.8
      \begin{itemize}
        \item Package: \texttt{/opt/apps/git/1.8}
        \item Modulefile: \texttt{{\color{blue}\$M}/Core/git/1.8}
      \end{itemize}
    \item Intel compilers 11.1
      \begin{itemize}
        \item Package: \texttt{/opt/apps/intel/17.0}
        \item Modulefile: \texttt{{\color{blue}\$M}/Core/intel/17.0}
        \item Modulefile adds \texttt{{\color{blue}\$M}/Compiler/intel/17.0} to \texttt{MODULEPATH}
      \end{itemize}
  \end{itemize}
\end{frame}



\begin{frame}{Hierarchical Examples: Compiler Dependent}
  \begin{itemize}
    \item Generic:
      \begin{itemize}
        \item Package: /opt/apps/\emph{compiler-version/package/version}
        \item M: {\color{blue}/opt/apps/modulefiles}
        \item Modulefile: \texttt{{\color{blue}\$M}/Compiler/}\emph{compiler/version/package/version}
      \end{itemize}
    \item Openmpi 1.8 with gcc 6.3
      \begin{itemize}
        \item Package: \texttt{/opt/apps/gcc-6\_3/openmpi/1.8}
        \item Modulefile: \texttt{{\color{blue}\$M}/Compiler/gcc/6.3/openmpi/1.8}
        \item Modulefile adds \texttt{{\color{blue}\$M}/MPI/gcc/6.3/openmpi/1.8}
          to \texttt{MODULEPATH} \\
      \end{itemize}
    \item Openmpi 1.8 with intel 11.1
      \begin{itemize}
        \item Package: \texttt{/opt/apps/intel-17\_0/openmpi/1.8}
        \item Modulefile: \texttt{{\color{blue}\$M}/Compiler/intel/17.0/openmpi/1.8}
        \item Modulefile adds \texttt{\$M/MPI/intel/17.0/openmpi/1.8}
          to \texttt{MODULEPATH}

      \end{itemize}
  \end{itemize}
\end{frame}

\begin{frame}{Hierarchical Examples: MPI/Compiler Dependent}
  \begin{itemize}
    \item PETSc 4.1 (1)
      \begin{itemize}
        \item Package: \texttt{/opt/apps/intel-17\_0/openmpi-1\_8/petsc/4.1}
        \item Modulefile: \texttt{{\color{blue}\$M}/MPI/intel/17.0/openmpi/1.8/petsc/4.1}
      \end{itemize}
    \item PETSc 4.1 (2)
      \begin{itemize}
        \item Package: \texttt{/opt/apps/gcc-4\_5/mvapich2-2\_1/petsc/4.1}
        \item Modulefile: \texttt{{\color{blue}\$M}/MPI/gcc/6.3/mvapich2/2.1/petsc/4.1}
      \end{itemize}
  \end{itemize}
\end{frame}


\begin{frame}{Loading the correct module}
  \begin{itemize}
    \item User loads ``\texttt{intel/17.0}'' module
    \item Can only see/load compiler dependent packages that are built with
      intel 17.0 compiler.
    \item Can not see/load package built with other versions or other compilers.
    \item Similar loading ``\texttt{openmpi/1.8'' module.
    \item User can only load package that are built w/ intel 17.0 and
      openmpi 1.8 and no others.
  \end{itemize}
\end{frame}








%page 8
\begin{frame}{Module Architecture Design at TACC}
  \begin{itemize}
    \item Be consistent! Find a convention and stick with it!
    \item We define consistent variables in each module:
      $<$SITE\_NAME$>$\_$<$PKG\_Name$>$\_\{LIB,INC,BIN\}
      \begin{itemize}
        \item TACC\_HDF5\_INC
        \item TACC\_HDF5\_BIN
      \end{itemize}
  \end{itemize}
\end{frame}

%%page 
%\begin{frame}[fragile]
%    \frametitle{}
%  \begin{itemize}
%    \item 
%  {\small
%    \begin{alltt}
%
%    \end{alltt}
%}
%    \item 
%  \end{itemize}
%\end{frame}





\begin{frame}{Conclusions: Lmod 7+}
  \begin{itemize}
    \item Latest version: https://github.com:TACC/Lmod.git
    \item Stable version: http://lmod.sf.net
    \item Documentation:  http://lmod.readthedocs.org
    \item Mailing List:   lmod-users@lists.sourceforge.net.
    \item Join here: https://lists.sourceforge.net/lists/listinfo/lmod-users
  \end{itemize}
\end{frame}



\begin{frame}{License}
\footnotesize
\textcopyright The University of Texas at Austin, \the\year\\
\vspace{0.5 cm}
This work is licensed under the Creative Commons Attribution Non-Commercial 3.0 Unported License. To view a copy of this license, 
visit http://creativecommons.org/licenses/by-nc/3.0/\\
\vspace{0.5 cm}
When attributing this work, please use the following text: 
"Designing and Administering Large-scale Systems", Texas Advanced Computing Center, \the\year. Available under a Creative Commons Attribution Non-Commercial 3.0 Unported License.
\newline
\newline
\includegraphics[width=1cm]{./figures/cc.png}
\end{frame}

\end{document}












