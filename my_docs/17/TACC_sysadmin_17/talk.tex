\documentclass[dvipsnames,aspectratio=169]{beamer}

\usepackage{etex}
\usepackage{tikz}
\usetikzlibrary{shapes.geometric, arrows}
\usetikzlibrary{decorations.pathreplacing}
\usetikzlibrary{calc,arrows.meta,positioning}
\usepackage{pstricks}
\usepackage{alltt}
\usepackage{colortbl}
\usepackage{textcomp}
\usepackage{picture}
\usepackage{outlines}
\usepackage{listings}

\usepackage{./themes/beamerthemeTACC16-169}


\begin{document}
\title[Lmod]{Lmod: A Modern Environment System}
\author{Robert McLay} 
\date{April 20, 2017} 


% page 0
\frame{\titlepage} 

% page 1
\begin{frame}{Outline}
  \begin{itemize}
    \item What are Environment Modules?
    \item What is Lmod?
    \item Modulefiles Architecture
    \item Where to go for help.
  \end{itemize}
\end{frame}

% page 2
\begin{frame}{Conclusions: Lmod}
  \begin{itemize}
    \item Latest version: https://github.com:TACC/Lmod.git
    \item Stable version: http://lmod.sf.net
    \item Documentation:  http://lmod.readthedocs.org
    \item Mailing List:   lmod-users@lists.sourceforge.net.
    \item Join here: https://lists.sourceforge.net/lists/listinfo/lmod-users
  \end{itemize}
\end{frame}

% page 3
\begin{frame}{What are Modules?}
  \begin{itemize}
    \item Modules are the way sites provide optional software: MPI,
      Boost, ...
    \item Modules add to PATH and set other env. vars for each package
    \item Modules can be unloaded.
    \item Modulefiles are in one file not one for each
      shell. (e.g. Compiler init scripts)
    \item Demo
  \end{itemize}
\end{frame}

% page 4
\begin{frame}{What is Lmod?}
  \begin{itemize}
    \item A modern replacement for a tried and true concept.
    \item The guiding principal: ``Make life easier w/o getting in
      the way.''
    \item Reads both TCL and Lua modulefiles
  \end{itemize}
\end{frame}

% page 5
\begin{frame}{Fundamental Issues}
  \begin{itemize}
    \item Software Packages are created and updated all the time.
    \item Some Users need new versions for new features and bug fixes.
    \item Other Users need older versions for stability and continuity.
    \item No system can support all versions of all packages.
    \item User programs using pre-built C++ \& Fortran libraries must
      link with the same compiler.
    \item Similarly, MPI Applications must build and link with same
      MPI/Compiler pairing when using prebuilt MPI libraries.
  \end{itemize}
\end{frame}

% page 6
\begin{frame}[fragile]
    \frametitle{Example of Lmod: Environment Modules (I)}
    {\tiny
\begin{alltt}
\$ {\color{blue} module avail}
------------------ /opt/apps/modulefiles/MPI/intel/12.0/mpich2/1.4 ------------------
  petsc/3.1 (default)    petsc/3.1-debug    pmetis/4.0    tau/2.20.3

------------------- /opt/apps/modulefiles/Compiler/intel/12.0 -----------------------
  boost/1.45.0              gotoblas2/1.13          openmpi/1.4.3
  boost/1.46.0              mpich2/1.3.2            openmpi/1.5.1
  boost/1.46.1 (default)    mpich2/1.4 (default)    openmpi/1.5.3 (default)

-------------------------- /opt/apps/modulefiles/Core -------------------------------
  StdEnv               intel/11.1               papi/4.1.4
  admin/admin-1.0      intel/12.0 (default)     scite/2.28
  ddt/ddt              lmod/lmod                tex/2010
  dmalloc/dmalloc      local/local (default)    unix/unix (default)
  fdepend/1.2          mkl/mkl                  visit/visit
  gcc/4.4              noweb/2.11b
  gcc/4.5 (default)
\end{alltt}
    }
\end{frame}

% page 7
\begin{frame}[fragile]
    \frametitle{Example of Lmod: Environment Modules (II)}
    {\tiny
\begin{alltt}
{\color{blue}\$ module list}
Currently Loaded Modules:
  1) StdEnv  2) gcc/4.5  3) mpich2/1.4  4) petsc/3.1
{\color{blue}\$ module unload gcc}
Inactive Modules:
  1) mpich2  2) petsc
{\color{blue}\$ module list}
Currently Loaded Modules:
  1) StdEnv
Inactive Modules:
  1) mpich2  2) petsc
{\color{blue}\$ module load intel}
Activating Modules:
  1) mpich2  2) petsc
{\color{blue}\$ module swap intel gcc}
Due to MODULEPATH changes the follow modules have been reloaded:
  1) mpich2  2) petsc
\end{alltt}
    }
\end{frame}

% page 8
\begin{frame}{Why You Might Want To Use Lmod?}
  \begin{itemize}
    \item Same \texttt{module} command as in Tmod
    \item Active Development;  Frequent Releases; Bug fixes.
    \item Vibrant Community
    \item It is used from Norway to Isreal to New Zealand from Stanford to MIT to NASA
    \item Enjoy many capabilities w/o changing a single module file
    \item Debian, Fedora and Brew packages available
    \item Many more advantages when you're ready
    \item It is what we and many sites around the world use every day!
  \end{itemize}
\end{frame}

% page 9
\begin{frame}{Features}
  \begin{itemize}
    \item Reads for TCL and Lua modulefiles
    \item One name rule.
    \item Support a Software Hierarchy
    \item Fast \texttt{module avail} via optional spider cache 
    \item Properties (gpu, mic)
    \item Semantic Versioning:  5.6 is older than 5.10
    \item family(``compiler'') family(``mpi'') support
    \item Optional Tracking: What modules are used?
    \item Many other features: ml, collections, hooks, nag, ...
  \end{itemize}
\end{frame}

% page 10
\begin{frame}{Tmod vs. Lmod}
  \begin{itemize}
    \item Tmod is in maintenance mode, Lmod active
    \item Lmod has many more features
    \item Tmod: \texttt{module load gcc/5.3 gcc/6.0} works
    \item Lmod has the ``One Name Rule''
    \item Lmod close to Tmod, but not the same.
  \end{itemize}
\end{frame}

%page 11
\begin{frame}{Module Architecture Design}
  \begin{itemize}
    \item Lua or TCL modulefiles?
    \item Optional software: shared or local?
    \item Flat or Hierarchical Module Layout?
    \item Naming conventions? (N/V, C/N/V, C/N/V/V?)
    \item Type of User: Expert, Novice, M\&S, Other?
    \item Keep software for life of cluster or not?
    \item Problems with Bash
  \end{itemize}
\end{frame}

%page 12
\begin{frame}{Shared Disk versus Local Install}
  \begin{itemize}
    \item Local install: Fast, small 
    \item Shared Disk: Big, slow
    \item TACC: local install (mostly)
    \item Limited to 3 version max!
    \item Shared Disk: Keep Software for life of system.
  \end{itemize}
\end{frame}

%page 13
\begin{frame}{Modulefile Choices}
  \begin{itemize}
    \item Flat Naming Scheme
    \item Hierarchical Naming Scheme
  \end{itemize}
\end{frame}

%page 14
\begin{frame}{Flat Naming Scheme: PETSc}
  PETSc is a parallel iterative solver package:
  \begin{itemize}
    \item Compilers: GCC 6.3, Intel 17.0
    \item MPI Implementations: MVAPICH2 2.1, IMPI 17.0
    \item MPI Solver package: PETSc 4.1
    \item 4 versions of PETSc: 2 Compilers $\times$ 2 MPI
  \end{itemize}
\end{frame}

%page 15
\begin{frame}{Flat: PETSc }
  \begin{enumerate}
  \item \texttt{PETSc/mvapich2\_2.1\_gcc\_6.3-4.1}
  \item \texttt{PETSc/mvapich2\_2.1\_intel\_17.0-4.1}
  \item \texttt{PETSc/openmpi\_1.8\_gcc\_6.3-4.1}
  \item \texttt{PETSc/openmpi\_1.8\_intel\_17.0-4.1}
  \end{enumerate}
\end{frame}

%page 16
\begin{frame}{Problems w/ Flat naming scheme}
  \begin{itemize}
    \item Users have to load modules:
      \begin{itemize}
        \item ``intel/17.0''
        \item ``mvapich2/intel\_17.0-2.1''
        \item ``PETSc/mvapich2\_2.1\_intel\_17.0-4.1''
        \item Changing compilers means unloading all three modules
        \item Reloading new compiler, MPI, PETSc modules.
        \item Not loading correct modules $\Rightarrow$ Mysterious Failures!
        \item Onus of package compatibility on users!
        \item Or extremely compilicated modulefiles!
      \end{itemize}
  \end{itemize}
\end{frame}

%page 17
\begin{frame}{Extremely compilicated modulefiles}
  \begin{itemize}
    \item Protect users via conflicts and/or prereqs
    \item The problem is that they are fragile
    \item What happens with a new compiler or MPI stack?
  \end{itemize}
\end{frame}

%page 18
\begin{frame}{Hierarchical Naming Schemes}
  \begin{itemize}
    \item Store modules under one tree: \texttt{/opt/apps/modulefiles}
    \item One strategy is to use sub-directories:
      \begin{itemize}
        \item Core: Regular packages: apps, compilers, git
        \item Compiler: Packages that depend on compiler: boost, MPI
        \item MPI: Packages that depend on MPI/Compiler: PETSc, TAU
      \end{itemize}
  \end{itemize}
\end{frame}

%page 19
\begin{frame}{\texttt{MODULEPATH}}
  \begin{itemize}
    \item \texttt{MODULEPATH} is a colon separated list of directories
      containing directories and module files.
    \item No modulefiles loaded $\Rightarrow$ users can only load core modules.
    \item Loading a compiler module adds to the \texttt{MODULEPATH}:
      \begin{itemize}
        \item Users can load compiler dependent modules.
        \item This includes MPI implementations modules.
      \end{itemize}
    \item Loading an MPI module adds to the \texttt{MODULEPATH}:
      \begin{itemize}
        \item Users can load MPI libraries that match the MPI/compiler pairing.
      \end{itemize}
  \end{itemize}
\end{frame}


%page 20
\begin{frame}{Hierarchical Examples: Core}
  \begin{itemize}
    \item Generic:
      \begin{itemize}
        \item Package: \texttt{/opt/apps/}\emph{package/version}
        \item M: {\color{blue}/opt/apps/modulefiles}
        \item Modulefile: \texttt{{\color{blue}\$M}/Core/}\emph{package/version}
      \end{itemize}
    \item Git 1.8
      \begin{itemize}
        \item Package: \texttt{/opt/apps/git/1.8}
        \item Modulefile: \texttt{{\color{blue}\$M}/Core/git/1.8}
      \end{itemize}
    \item Intel compilers 11.1
      \begin{itemize}
        \item Package: \texttt{/opt/apps/intel/17.0}
        \item Modulefile: \texttt{{\color{blue}\$M}/Core/intel/17.0}
        \item Modulefile adds \texttt{{\color{blue}\$M}/Compiler/intel/17.0} to \texttt{MODULEPATH}
      \end{itemize}
  \end{itemize}
\end{frame}



%page 21
\begin{frame}{Hierarchical Examples: Compiler Dependent}
  \begin{itemize}
    \item Generic:
      \begin{itemize}
        \item Package: /opt/apps/\emph{compiler-version/package/version}
        \item M: {\color{blue}/opt/apps/modulefiles}
        \item Modulefile: \texttt{{\color{blue}\$M}/Compiler/}\emph{compiler/version/package/version}
      \end{itemize}
    \item Openmpi 1.8 with gcc 6.3
      \begin{itemize}
        \item Package: \texttt{/opt/apps/gcc-6\_3/openmpi/1.8}
        \item Modulefile: \texttt{{\color{blue}\$M}/Compiler/gcc/6.3/openmpi/1.8}
        \item Modulefile adds \texttt{{\color{blue}\$M}/MPI/gcc/6.3/openmpi/1.8}
          to \texttt{MODULEPATH} \\
      \end{itemize}
    \item Openmpi 1.8 with intel 11.1
      \begin{itemize}
        \item Package: \texttt{/opt/apps/intel-17\_0/openmpi/1.8}
        \item Modulefile: \texttt{{\color{blue}\$M}/Compiler/intel/17.0/openmpi/1.8}
        \item Modulefile adds \texttt{\$M/MPI/intel/17.0/openmpi/1.8}
          to \texttt{MODULEPATH}

      \end{itemize}
  \end{itemize}
\end{frame}

%page 22
\begin{frame}{Hierarchical Examples: MPI/Compiler Dependent}
  \begin{itemize}
    \item PETSc 4.1 (1)
      \begin{itemize}
        \item Package: \texttt{/opt/apps/intel-17\_0/openmpi-1\_8/petsc/4.1}
        \item Modulefile: \texttt{{\color{blue}\$M}/MPI/intel/17.0/openmpi/1.8/petsc/4.1}
      \end{itemize}
    \item PETSc 4.1 (2)
      \begin{itemize}
        \item Package: \texttt{/opt/apps/gcc-4\_5/mvapich2-2\_1/petsc/4.1}
        \item Modulefile: \texttt{{\color{blue}\$M}/MPI/gcc/6.3/mvapich2/2.1/petsc/4.1}
      \end{itemize}
  \end{itemize}
\end{frame}


%page 23
\begin{frame}{Loading the correct module}
  \begin{itemize}
    \item User loads ``\texttt{intel/17.0}'' module
    \item Can only see/load compiler dependent packages that are built with
      intel 17.0 compiler.
    \item Can not see/load package built with other versions or other compilers.
    \item Similar loading ``\texttt{openmpi/1.8}'' module.
    \item User can only load package that are built w/ intel 17.0 and
      openmpi 1.8 and no others.
  \end{itemize}
\end{frame}


%page 24
\begin{frame}{Modulefile contents}
  \begin{itemize}
    \item Be consistent! Find a convention and stick with it!
    \item We define consistent variables in each module:
      $<$SITE\_NAME$>$\_$<$PKG\_Name$>$\_\{LIB,INC,BIN\}
      \begin{itemize}
        \item TACC\_HDF5\_INC
        \item TACC\_HDF5\_BIN
      \end{itemize}
  \end{itemize}
\end{frame}

%%page 
%\begin{frame}[fragile]
%    \frametitle{}
%  \begin{itemize}
%    \item 
%  {\small
%    \begin{alltt}
%
%    \end{alltt}
%}
%    \item 
%  \end{itemize}
%\end{frame}

%page 25
\begin{frame}{Module Naming Conventions}
  \begin{itemize}
    \item N/V:   Name/Version (e.g. \texttt{bowtie/2.3})
    \item C/N/V: Category/Name/version (e.g. \texttt{bio/bowtie/2.3})
    \item N/V/V: Category/Sub/Name/version (e.g. \texttt{/bowtie/64/2.3})
  \end{itemize}
\end{frame}

%page 26
\begin{frame}{Module Naming Conventions (II)}
  \begin{itemize}
    \item Try to stick with N/V if possible
    \item It's less typing
    \item C/N/V might be helpful to novice users
    \item But your obvious categories may not be obvious to your users.
    \item Avoid N/V/V unless your users are experts 
    \item Or if you really need 64/32 bit libraries
  \end{itemize}
\end{frame}

%page 27
\begin{frame}{Bash Issues}
  \begin{itemize}
    \item Bash Startup is typically ``broken'' for non-login
      interactive shells
    \item Redhat, Centos, MacOS typically don't source /etc/bashrc on
      interactive shells
    \item MPI jobs start an interactive shell.
  \end{itemize}
\end{frame}

%page 28
\begin{frame}{Bash Issues (II)}
  \begin{itemize}
    \item Want module command to work in all shells.
    \item Want stacksize unlimited for MPI jobs
    \item We patched bash to force it to source /etc/tacc/bashrc
  \end{itemize}
\end{frame}

%page 29
\begin{frame}{Bash Repair Choices}
  \begin{itemize}
    \item Switch users to Z shell?
    \item patch bash (see Lmod docs)
    \item Expect all users to source /etc/bashrc in \textasciitilde/.bashrc
    \item Expect all users to start jobs with \#!/bin/bash -l
  \end{itemize}
\end{frame}

%page 30
\begin{frame}{Keeping Software for life of Machine or Not}
  \begin{itemize}
    \item It is possible with a shared disk approach
    \item You might want to hide older modules
  \end{itemize}
\end{frame}

%page 31
\begin{frame}{Hidden Modules}
  \begin{itemize}
    \item Sites have always hide modules by adding a leading dot
    \item For example gcc/.6.3
    \item Lmod 7 also allows for hidden module via MODULERC
    \item system MODULERC or ~/.modulerc
  \end{itemize}
\end{frame}

%page 32
\begin{frame}[fragile]
    \frametitle{Using System MODULERC to hide modules}
  \begin{itemize}
    \item In \$MODULERCFILE
  {\small
    \begin{alltt}
        #%Module
        hide-version foo/3.2
    \end{alltt}
}
  \end{itemize}
\end{frame}

%page 33
\begin{frame}{Deprecating Modules}
  \begin{itemize}
    \item Sometime you may have to remove a package (and its
      modulefile)
    \item Certainly with Local installs.
    \item How do you let users know?
  \end{itemize}
\end{frame}

%page 34
\begin{frame}{Deprecating Modules (II)}
  \begin{itemize}
    \item Assume Lmod is installed as /opt/apps/lmod/7.4.1/...
    \item Then /opt/apps/lmod/etc/admin.list is the ``nag'' file
    \item Or set \$LMOD\_ADMIN\_FILE to any file.
    \item Every time a particular modulefile is loaded the user gets a
      message.
    \item We have found that setting a nag message to very effective!
  \end{itemize}
\end{frame}

%page 35
\begin{frame}{Tracking Module Usage}
  \begin{itemize}
    \item Lmod makes it easy to track module usage.
    \item Lmod can be setup to send a tagged message to syslog
    \item Rsyslog can send tags to a separate file.
    \item See lmod/contrib/tracking\_module\_usage for details
  \end{itemize}
\end{frame}

%page 36
\begin{frame}[fragile]
    \frametitle{Usage counts}
  {\small
    \begin{alltt}
    \$ analyzeLmodDB --sqlPattern '%fftw%' --start '2015-01-01 --end '2015-02-01'  counts

        Module path                                   Distinct Users
        -----------                                   --------------  
        /apps/intel13/mvapich2\_1\_9/mfiles/fftw3/3.3.2             151
        /apps/intel13/mvapich2\_1\_9/mfiles/fftw2/2.1.5              62
        /apps/intel13/impi\_4\_1/mfiles/fftw3/3.3.2                  45
        /apps/intel13/impi\_4\_1/mfiles/fftw2/2.1.5                  19

    \end{alltt}
}
\end{frame}

%page 37
\begin{frame}[fragile]
    \frametitle{Distinct Users}
  {\small
    \begin{alltt}
     \$ ./analyzeLmodDB --sqlPattern '%/apps/modulefiles/settarg%' usernames

     Module path                       User Name
     -----------                       ---------
     /opt/apps/mfiles/settarg/5.8      user1
     /opt/apps/mfiles/settarg/5.8      user2
     /opt/apps/mfiles/settarg/5.8      user3
     /opt/apps/mfiles/settarg/5.8.1    mclay
     /opt/apps/mfiles/settarg/5.9.1    user5
    \end{alltt}
}
\end{frame}

%page 38
\begin{frame}{Why does Lmod work at all?}
  \begin{itemize}
    \item The Environment is inherited from the parent process
    \item Changes in the child's enviroment DOES NOT affect the
      parent's
    \item So how could Lmod work at all? 
  \end{itemize}
\end{frame}

%page 39
\begin{frame}{The trick is}
  \begin{itemize}
    \item The \texttt{lmod} program generates text.
    \item The module command eval's that text.
  \end{itemize}
\end{frame}

%page 40
\begin{frame}{Why is this important?}
  \begin{itemize}
    \item It's a useful trick to know
    \item Debugging Modulefiles:
    \item \texttt{\$LMOD\_CMD bash load} \emph{module} \texttt{2$>$
        /dev/null $>$ stdout.txt}
  \end{itemize}
\end{frame}

%page 41
\begin{frame}{Debugging Lmod}
  \begin{itemize}
    \item \texttt{module --config} : reports Lmod configuration
    \item \texttt{module -D load foo $>$ load.log}
  \end{itemize}
\end{frame}

%page 42
\begin{frame}{Tracing Lmod}
  \begin{itemize}
    \item A new feature of Lmod 7.4.4+
    \item module -T ...
    \item Can trace loads and how restores work.
  \end{itemize}
\end{frame}

%page 43
\begin{frame}[fragile]
    \frametitle{Tracing Example}
  \begin{itemize}
    \item 
  {\small
    \begin{alltt}
running: module --initial_load restore
  Using collection:      /home/user/.lmod.d/default
  Setting MODULEPATH to: /opt/apps/modulefiles/Darwin:/opt/apps/modulefiles/Core
  Loading: unix (fn: /opt/apps/modulefiles/Core/unix/unix.lua)
  Loading: local (fn: /opt/apps/modulefiles/Core/local/local.lua)
  Loading: mkl (fn: /opt/apps/modulefiles/Core/mkl/mkl.lua)
  Loading: gcc (fn: /opt/apps/modulefiles/Darwin/gcc/5.2.lua)
  Loading: StdEnv (fn: /opt/apps/modulefiles/Core/StdEnv.lua)
    \end{alltt}
}
    \item 
  \end{itemize}
\end{frame}







\begin{frame}{Conclusions: Lmod}
  \begin{itemize}
    \item Latest version: https://github.com:TACC/Lmod.git
    \item Stable version: http://lmod.sf.net
    \item Documentation:  http://lmod.readthedocs.org
    \item Mailing List:   lmod-users@lists.sourceforge.net.
    \item Join here: https://lists.sourceforge.net/lists/listinfo/lmod-users
  \end{itemize}
\end{frame}


%
\begin{frame}{License}
\footnotesize
\textcopyright The University of Texas at Austin, \the\year\\
\vspace{0.5 cm}
This work is licensed under the Creative Commons Attribution Non-Commercial 3.0 Unported License. To view a copy of this license, 
visit http://creativecommons.org/licenses/by-nc/3.0/\\
\vspace{0.5 cm}
When attributing this work, please use the following text: 
"Designing and Administering Large-scale Systems", Texas Advanced Computing Center, \the\year. Available under a Creative Commons Attribution Non-Commercial 3.0 Unported License.
\newline
\newline
\includegraphics[width=1cm]{./figures/cc.png}
\end{frame}

\end{document}
